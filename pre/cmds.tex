%----------------------------------------------------------------------------------%
% Numerical sets                                                                   %
%----------------------------------------------------------------------------------%
\newcommand{\naturals}{\mathbb{N}}                                                 %
\newcommand{\integers}{\mathbb{Z}}                                                 %
\newcommand{\rationals}{\mathbb{Q}}                                                %
\newcommand{\reals}{\mathbb{R}}                                                    %
\newcommand{\complexfield}{\mathbb{C}}                                             %
%----------------------------------------------------------------------------------%
% Spaces of constant sectional curvature                                           %
%----------------------------------------------------------------------------------%
\newcommand{\euclideanspace}[1]{\mathbb{R}^{#1}}                                   %
\newcommand{\hyperbolicspace}[1]{\mathbb{H}^{#1}}                                  %
\newcommand{\roundsphere}[1]{\mathbb{S}^{#1}}                                      %
%----------------------------------------------------------------------------------%
% Absolute value                                                                   %
%----------------------------------------------------------------------------------%
\DeclarePairedDelimiter\abs{\lvert}{\rvert}                                        %
%----------------------------------------------------------------------------------%
% Norm and, more generally, p-norm                                                 %
%----------------------------------------------------------------------------------%
% \norm                                                                            %
%----------------------------------------------------------------------------------%
% Examples:                                                                        %
%----------------------------------------------------------------------------------%
%  - \norm          --> \lvert{\cdot}\rvert                                        %
%  - \norm<n>       --> \lvert{\cdot}\rvert^{n}                                    %
%  - \norm[p]       --> \lvert{\cdot}\rvert_{p}                                    %
%  - \norm<n>[p]    --> \lvert{\cdot}\rvert_{p}^{n}                                %
%  - \norm(x)       --> \lvert{x}\rvert                                            %
%  - \norm<n>(x)    --> \lvert{x}\rvert^{n}                                        %
%  - \norm[p](x)    --> \lvert{x}\rvert_{p}                                        %
%  - \norm<n>[p](x) --> \lvert{x}\rvert_{p}^{n}                                    %
%----------------------------------------------------------------------------------%
\NewDocumentCommand{\norm}{d <> d [] d ()}{                                        %
  \IfValueTF{#1}{                                                                  %
    \IfValueTF{#2}{                                                                %
      \IfValueTF{#3}{                                                              %
        \lVert{#3}\rVert_{#2}^{#1}                                                 %
      }{                                                                           %
        \lVert{\cdot}\rVert_{#2}^{#1}                                              %
      }                                                                            %
    }{                                                                             %
      \IfValueTF{#3}{                                                              %
        \lVert{#3}\rVert^{#1}                                                      %
      }{                                                                           %
        \lVert{\cdot}\rVert^{#1}                                                   %
      }                                                                            %
    }                                                                              %
  }{                                                                               %
    \IfValueTF{#2}{                                                                %
      \IfValueTF{#3}{                                                              %
        \lVert{#3}\rVert_{#2}                                                      %
      }{                                                                           %
        \lVert{\cdot}\rVert_{#2}                                                   %
      }                                                                            %
    }{                                                                             %
      \IfValueTF{#3}{                                                              %
        \lVert{#3}\rVert\                                                          %
      }{                                                                           %
        \lVert{\cdot}\rVert\                                                       %
      }                                                                            %
    }                                                                              %
  }                                                                                %
}                                                                                  %
%----------------------------------------------------------------------------------%
% \innerproduct{ vector_1, vector_2 }                                              %
%----------------------------------------------------------------------------------%
\DeclarePairedDelimiter\innerproduct{\langle}{\rangle}                             %
%----------------------------------------------------------------------------------%
% FUNCTION SPACES                                                                  %
%----------------------------------------------------------------------------------%
% \smoothfunctions                                                                 %
%----------------------------------------------------------------------------------%
% Examples:                                                                        %
%----------------------------------------------------------------------------------%
%  - \smoothfunctions    --> \C^{\infty}                                           %
%  - \smoothfunctions(M) --> \C^{\infty}\left(M\right)                             %
%----------------------------------------------------------------------------------%
\NewDocumentCommand{\smoothfunctions}{d ()}{                                       %
  \IfValueTF{#1}{                                                                  %
    C^{\infty}\left(#1\right)                                                      %
  }{                                                                               %
    C^{\infty}                                                                     %
  }                                                                                %
}                                                                                  %
%----------------------------------------------------------------------------------%
% \smoothvectorfields                                                              %
%----------------------------------------------------------------------------------%
\NewDocumentCommand{\smoothvectorfields}{d ()}{                                    %
  \IfValueTF{#1}{                                                                  %
    \mathfrak{X}^{\infty}\left(#1\right)                                           %
  }{                                                                               %
    \mathfrak{X}^{\infty}                                                          %
  }                                                                                %
}                                                                                  %
%----------------------------------------------------------------------------------%
% \diffeomorphisms                                                                 %
%----------------------------------------------------------------------------------%
% Examples:                                                                        %
%----------------------------------------------------------------------------------%
%  - \diffeomorphisms    --> \mathcal{D}                                           %
%  - \diffeomorphisms(M) --> \mathcal{D}\left(M\right)                             %
%----------------------------------------------------------------------------------%
\NewDocumentCommand{\diffeomorphisms}{d ()}{                                       %
  \IfValueTF{#1}{                                                                  %
    \mathcal{D}\left(#1\right)                                                     %
  }{                                                                               %
    \mathcal{D}                                                                    %
  }                                                                                %
}                                                                                  %
%----------------------------------------------------------------------------------%
% \riemannianmetrics                                                               %
%----------------------------------------------------------------------------------%
% Examples:                                                                        %
%----------------------------------------------------------------------------------%
%  - \riemannianmetrics    --> \mathcal{M}                                         %
%  - \riemannianmetrics(M) --> \mathcal{M}\left(M\right)                           %
%----------------------------------------------------------------------------------%
\NewDocumentCommand{\riemannianmetrics}{d ()}{                                     %
  \IfValueTF{#1}{                                                                  %
    \mathcal{M}\left(#1\right)                                                     %
  }{                                                                               %
    \mathcal{M}                                                                    %
  }                                                                                %
}                                                                                  %
%----------------------------------------------------------------------------------%
% DIFFERENTIAL OPERATORS                                                           %
%----------------------------------------------------------------------------------%
% \gradientfield                                                                   %
%----------------------------------------------------------------------------------%
% Examples:                                                                        %
%----------------------------------------------------------------------------------%
%  - \gradientfield{f}    --> \textup{grad}\left(f\right)                          %
%  - \gradientfield<g>{f} --> \textup{grad}_{g}\left(f\right)                      %
%----------------------------------------------------------------------------------%
\NewDocumentCommand{\gradientfield}{d <> m}{                                       %
  \IfValueTF{#1}{                                                                  %
    \textup{grad}_{#1}\left(#2\right)                                              %
  }{                                                                               %
    \textup{grad}\left(#2\right)                                                   %
  }                                                                                %
}                                                                                  %
%----------------------------------------------------------------------------------%
% \hessian                                                                         %
%----------------------------------------------------------------------------------%
% Examples:                                                                        %
%----------------------------------------------------------------------------------%
% - \hessian{f}          --> \textup{Hess}\left(f\right)                           %
% - \hessian<g>{f}       --> \textup{Hess}_{g}\left(f\right)                       %
% - \hessian{f}[X]       --> \textup{Hess}\left(f\right)\left(X,\cdot\right)       %
% - \hessian<g>{f}[X]    --> \textup{Hess}_{g}\left(f\right)\left(X,\cdot\right)   %
% - \hessian{f}(Y)       --> \textup{Hess}\left(f\right)\left(\cdot,Y\right)       %
% - \hessian<g>{f}(Y)    --> \textup{Hess}_{g}\left(f\right)\left(\cdot,Y\right)   %
% - \hessian{f}[X](Y)    --> \textup{Hess}\left(f\right)\left(X,Y\right)           %
% - \hessian<g>{f}[X](Y) --> \textup{Hess}_{g}\left(f\right)\left(X,Y\right)       %
%----------------------------------------------------------------------------------%
\NewDocumentCommand{\hessian}{d <> m d [] d ()}{                                   %
  \IfValueTF{#1}{                                                                  %
    \IfValueTF{#3}{                                                                %
      \IfValueTF{#4}{                                                              %
        \textup{Hess}_{#1}\left(#2\right)\left(#3,#4\right)                        %
      }{                                                                           %
        \textup{Hess}_{#1}\left(#2\right)\left(#3,\cdot\right)                     %
      }                                                                            %
    }{                                                                             %
      \IfValueTF{#4}{                                                              %
        \textup{Hess}_{#1}\left(#2\right)\left(\cdot,#4\right)                     %
      }{                                                                           %
        \textup{Hess}_{#1}\left(#2\right)                                          %
      }                                                                            %
    }                                                                              %
  }{                                                                               %
    \IfValueTF{#3}{                                                                %
      \IfValueTF{#4}{                                                              %
        \textup{Hess}\left(#2\right)\left(#3,#4\right)                             %
      }{                                                                           %
        \textup{Hess}\left(#2\right)\left(#3,\cdot\right)                          %
      }                                                                            %
    }{                                                                             %
      \IfValueTF{#4}{                                                              %
        \textup{Hess}\left(#2\right)\left(\cdot,#4\right)                          %
      }{                                                                           %
        \textup{Hess}\left(#2\right)                                               %
      }                                                                            %
    }                                                                              %
  }                                                                                %
}                                                                                  %
%----------------------------------------------------------------------------------%
% \laplacian                                                                       %
%----------------------------------------------------------------------------------%
% Examples:                                                                        %
%----------------------------------------------------------------------------------%
%  - \laplacian{f}    --> \Delta{f}                                                %
%  - \laplacian<g>{f} --> \left(\Delta{f}\right)_{g}                               %
%----------------------------------------------------------------------------------%
\NewDocumentCommand{\laplacian}{d <> m}{                                           %
  \IfValueTF{#1}{                                                                  %
    \left(\Delta{#2}\right)_{#1}                                                   %
  }{                                                                               %
    \Delta{#2}                                                                     %
  }                                                                                %
}                                                                                  %
%----------------------------------------------------------------------------------%
% \liederivative                                                                   %
%----------------------------------------------------------------------------------%
\newcommand{\liederivative}[2]{\mathcal{L}_{#2}{#1}}                               %
%----------------------------------------------------------------------------------%
% \curvaturetensor                                                                 %
%----------------------------------------------------------------------------------%
% Examples:                                                                        %
%----------------------------------------------------------------------------------%
%  - \curvaturetensor{X}{Y}       --> \textup{R}\left(X,Y\right)                   %
%  - \curvaturetensor<g>{X}{Y}    --> \textup{R}_{g}\left(X,Y\right)               %
%  - \curvaturetensor{X}{Y}(Z)    --> \textup{R}\left(X,Y\right)Z                  %
%  - \curvaturetensor<g>{X}{Y}(Z) --> \textup{R}_{g}\left(X,Y\right)Z              %
%----------------------------------------------------------------------------------%
\NewDocumentCommand{\curvaturetensor}{d <> m m d ()}{                              %
  \IfValueTF{#1}{                                                                  %
    \IfValueTF{#4}{                                                                %
      \textup{R}_{#1}\left(#2,#3\right)#4                                          %
    }{                                                                             %
      \textup{R}_{#1}\left(#2,#3\right)                                            %
    }                                                                              %
  }{                                                                               %
    \IfValueTF{#4}{                                                                %
      \textup{R}\left(#2,#3\right)#4                                               %
    }{                                                                             %
      \textup{R}\left(#2,#3\right)                                                 %
    }                                                                              %
  }                                                                                %
}                                                                                  %
%----------------------------------------------------------------------------------%
% \riccitensor                                                                     %
%----------------------------------------------------------------------------------%
% Examples:                                                                        %
%----------------------------------------------------------------------------------%
%  - \riccitensor          --> \textup{Rc}                                         %
%  - \riccitensor(X)       --> \textup{Rc}\left(X,\cdot\right)                     %
%  - \riccitensor[Y]       --> \textup{Rc}\left(\cdot,Y\right)                     %
%  - \riccitensor(X)[Y]    --> \textup{Rc}\left(X,Y\right)                         %
%  - \riccitensor<g>       --> \textup{Rc}_{g}                                     %
%  - \riccitensor<g>(X)    --> \textup{Rc}_{g}\left(X,\cdot\right)                 %
%  - \riccitensor<g>[Y]    --> \textup{Rc}_{g}\left(\cdot,Y\right)                 %
%  - \riccitensor<g>(X)[Y] --> \textup{Rc}_{g}\left(X,Y\right)                     %
%----------------------------------------------------------------------------------%
\NewDocumentCommand{\riccitensor}{d <> d [] d ()}{                                 %
  \IfValueTF{#1}{                                                                  %
    \IfValueTF{#2}{                                                                %
      \IfValueTF{#3}{                                                              %
        \textup{Rc}_{#1}\left(#2,#3\right)                                         %
      }{                                                                           %
        \textup{Rc}_{#1}\left(#2,\cdot\right)                                      %
      }                                                                            %
    }{                                                                             %
      \IfValueTF{#3}{                                                              %
        \textup{Rc}_{#1}\left(\cdot,#3\right)                                      %
      }{                                                                           %
        \textup{Rc}_{#1}                                                           %
      }                                                                            %
    }                                                                              %
  }{                                                                               %
    \IfValueTF{#2}{                                                                %
      \IfValueTF{#3}{                                                              %
        \textup{Rc}\left(#2,#3\right)                                              %
      }{                                                                           %
        \textup{Rc}\left(#2,\cdot\right)                                           %
      }                                                                            %
    }{                                                                             %
      \IfValueTF{#3}{                                                              %
        \textup{Rc}\left(\cdot,#3\right)                                           %
      }{                                                                           %
        \textup{Rc}                                                                %
      }                                                                            %
    }                                                                              %
  }                                                                                %
}                                                                                  %
%----------------------------------------------------------------------------------%
% \scalarcurvature                                                                 %
%----------------------------------------------------------------------------------%
% Examples:                                                                        %
%----------------------------------------------------------------------------------%
%  - \scalarcurvature    --> R                                                     %
%  - \scalarcurvature<g> --> R_{g}                                                 %
%----------------------------------------------------------------------------------%
\NewDocumentCommand{\scalarcurvature}{d <>}{                                       %
  \IfValueTF{#1}{                                                                  %
    \textup{R}_{#1}                                                                %
  }{                                                                               %
    \textup{R}                                                                     %
  }                                                                                %
}                                                                                  %
%----------------------------------------------------------------------------------%

% statistics
\newcommand{\samplemean}[1]{\bar{#1}}
\newcommand{\samplevariance}[1]{s_{#1}^{2}}
\newcommand{\samplestandarddeviation}[1]{s_{#1}}
