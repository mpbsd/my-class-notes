\section{Correção de um exercício da lista}\label{sec:correcao-de-um-exercício-da-lista}

\begin{exercise}
  The sample mean and sample variance of five data values are, respectively,
  \(\samplemean{x}=104\) and \(\samplevariance{x}=4\). If three of the data
  values are \(102,100,105\) what are the other two data values?
\end{exercise}

Suponha que
\(x_{1}^{0},x_{2}^{0},x_{3}^{0},\samplemean{x},\samplevariance{x}\in\reals\)
sejam constantes, onde \(\samplevariance{x}\geqslant{0}\), e que
\(x_{4},x_{5}\in\reals\) sejam números reais a serem encontrados de forma que
\[
  \left\{
    \begin{array}{rcl}
      \samplemean{x}
      &=&
      \dfrac{
        x_{1}^{0}+x_{2}^{0}+x_{3}^{0}+x_{4}+x_{5}
      }{5}
      \\
      &&
      \\
      \samplevariance{x}
      &=&
      \dfrac{
        (x_{1}^{0}-\samplemean{x})^{2}
        +
        (x_{2}^{0}-\samplemean{x})^{2}
        +
        (x_{3}^{0}-\samplemean{x})^{2}
        +
        (x_{4}-\samplemean{x})^{2}
        +
        (x_{5}-\samplemean{x})^{2}
      }{4}
    \end{array}
  \right.
\]
Observe, neste caso, que:
\begin{align}
  x_{4}+x_{5}
  &=
  5\samplemean{x}
  -
  \sum_{i=1}^{3}x_{i}^{0}
  \\
  (x_{4}-\samplemean{x})^{2}
  +
  (x_{5}-\samplemean{x})^{2}
  &=
  4\samplevariance{x}
  -
  \sum_{i=1}^{3}(x_{i}^{0}-\samplemean{x})^{2}
\end{align}
de onde segue que:
\[
  x_{4}^{2}+x_{5}^{2}
  -
  2\samplemean{x}(x_{4}+x_{5})
  +
  2\samplemean{x}^{2}
  =
  4\samplevariance{x}
  -
  \sum_{i=1}^{3}(x_{i}^{0})^{2}
  +
  2\samplemean{x}\sum_{i=1}^{3}x_{i}^{0}
  -
  3\samplemean{x}^{2}
  ,
\]
e, assim, que:
\[
  x_{4}^{2}+x_{5}^{2}
  -
  2
  \samplemean{x}
  \left(
    5\samplemean{x}
    -
    \sum_{i=1}^{3}x_{i}^{0}
  \right)
  +
  2\samplemean{x}^{2}
  =
  4\samplevariance{x}
  -
  \sum_{i=1}^{3}(x_{i}^{0})^{2}
  +
  2\samplemean{x}\sum_{i=1}^{3}x_{i}^{0}
  -
  3\samplemean{x}^{2}.
\]
Portanto, tem-se que:
\[
  5\samplemean{x}^{2}+4\samplevariance{x}-\sum_{i=1}^{3}(x_{i}^{0})^{2}
  =
  x_{4}^{2}+x_{5}^{2}
  \geqslant{0}.
\]
Isto mostra que os valores
\(x_{1}^{0},x_{2}^{0},x_{3}^{0},\samplemean{x},\samplestandarddeviation{x}\)
precisam satisfazer à relação:
\[
  \frac{\samplemean{x}^{2}}{a^{2}}
  +
  \frac{\samplevariance{x}}{b^{2}}
  \geqslant{1}
  ,
\]
onde
\[
  a=\frac{\sqrt{\sum_{i=1}^{3}(x_{i}^{0})^{2}}}{\sqrt{5}}
  \quad
  \text{e}
  \quad
  b=\frac{\sqrt{\sum_{i=1}^{3}(x_{i}^{0})^{2}}}{2}
  .
\]
Sabendo disso, você consegue agora escolher valores adequados?
=======
\section{Área do setor circular}\label{sec:area-do-setor-circular}

Para calcular a área do setor circular de uma circunferência, de raio \(R>0\),
determinada por um ângulo \(\theta\), \(0\leqslant{\theta}\leqslant{2\pi}\),
consideramos quatro casos:

\begin{enumerate}
  \item
    \(0\leqslant{\theta}\leqslant{\dfrac{\pi}{2}}\);
  \item
    \(\dfrac{\pi}{2}\leqslant{\theta}\leqslant{\pi}\);
  \item
    \(\pi\leqslant{\theta}\leqslant{\dfrac{3\pi}{2}}\) e, finalmente,
  \item
    \(\dfrac{3\pi}{2}\leqslant{\theta}\leqslant{2\pi}\).
\end{enumerate}

É suficiente provarmos que a área da primeira dentre as regiões acima listadas é
igual a \(\frac{R^{2}\theta}{2}\) pois, nos demais casos, teremos
respectivamente:

\begin{enumerate}
  \item
    \(\frac{R^{2}\pi}{4}+\frac{R^{2}\theta'}{2}=\frac{R^{2}}{2}\left(\frac{\pi}{2}+\theta'\right)=\frac{R^{2}\theta}{2}\),
    onde \(\frac{\pi}{2}\leqslant{\theta}\leqslant{\pi}\) e \(\theta'=\theta-\frac{\pi}{2}\);
  \item
    \(\frac{R^{2}\pi}{2}+\frac{R^{2}\theta'}{2}=\frac{R^{2}}{2}\left(\pi+\theta'\right)=\frac{R^{2}\theta}{2}\),
    onde \(\pi\leqslant{\theta}\leqslant{\frac{3\pi}{2}}\) e \(\theta'=\theta-\pi\);
  \item
    \(3\frac{R^{2}\pi}{4}+\frac{R^{2}\theta'}{2}=\frac{R^{2}}{2}\left(\frac{3\pi}{2}+\theta'\right)=\frac{R^{2}\theta}{2}\),
    onde \(\frac{3\pi}{2}\leqslant{\theta}\leqslant{2\pi}\) e \(\theta'=\theta-\frac{3\pi}{2}\).
\end{enumerate}

% Abaixo temos três exemplos, um para cada intervalo listado acima.

% \begin{figure}[H]
%   \centering
%   \begin{tikzpicture}
%     \pie[sum=360]{77/,283/}
%   \end{tikzpicture}
% \end{figure}

% \begin{figure}[H]
%   \centering
%   \begin{tikzpicture}
%     \pie[sum=360]{90/,37/,233/}
%   \end{tikzpicture}
% \end{figure}

% \begin{figure}[H]
%   \centering
%   \begin{tikzpicture}
%     \pie[sum=360]{180/,57/,123/}
%   \end{tikzpicture}
% \end{figure}

% \begin{figure}[H]
%   \centering
%   \begin{tikzpicture}
%     \pie[sum=360]{270/,27/,63/}
%   \end{tikzpicture}
% \end{figure}

Na tentativa de determinar a área da região de interesse chega-se à seguinte
soma de integrais:

\begin{align*}
  &\int_{0}^{R\cos{(\theta)}}x\tan{(\theta)}\,dx
  +
  \int_{R\cos{(\theta)}}^{R}\sqrt{R^{2}-x^{2}}\,dx
  =
  \\
  &=
  \left(
    \frac{x^{2}\tan{(\theta)}}{2}
  \right)
  \bigg{\vert}_{x=0}^{x=R\cos{(\theta)}}
    +
  \frac{R^{2}}{2}
  \left(
    \arcsin{\left(\frac{x}{R}\right)}
    +
    \frac{x\sqrt{R^{2}-x^{2}}}{R^{2}}
  \right)
  \Bigg{\vert}_{x=R\cos{(\theta)}}^{x=R}
  \\
  &=
  \frac{R^{2}\cos{(\theta)\sin{(\theta)}}}{2}
  \\
  &\qquad
  +
  \frac{R^{2}}{2}
  \left(
    \frac{\pi}{2}
    -
    \arcsin{\left(\frac{R\cos{(\theta)}}{R}\right)}
    -
    \frac{R\cos{(\theta)}\sqrt{R^{2}-R^{2}\cos^{2}{(\theta)}}}{R^{2}}
  \right)
  \\
  &=
  \frac{R^{2}\cos{(\theta)\sin{(\theta)}}}{2}
  +
  \frac{R^{2}}{2}
  \left(
    \frac{\pi}{2}
    -
    \arcsin{(\cos{(\theta)})}
    -
    \cos{(\theta)}\abs{\sin{(\theta)}}
  \right)
  \\
  &=
  \frac{R^{2}\cos{(\theta)\sin{(\theta)}}}{2}
  +
  \frac{R^{2}}{2}
  \left(
    \frac{\pi}{2}
    -
    \left(\frac{\pi}{2}-\theta\right)
    -
    \cos{(\theta)}\sin{(\theta)}
  \right)
  =
  \frac{R^{2}\theta}{2}
\end{align*}

onde utilizamos os seguintes fatos:

\begin{enumerate}
  \item
    \(\arcsin{(\cos{(\theta)})}=\frac{\pi}{2}-\theta\) pois
    \(\sin{(\frac{\pi}{2}-\theta)}=\cos{(\theta)}\) e, além disso, como
    \(0\leqslant{\theta}\leqslant{\frac{\pi}{2}}\), nós temos que
    \(0\leqslant{\frac{\pi}{2}-\theta}\leqslant{\frac{\pi}{2}}\);
  \item
    \(\abs{\sin{(\theta)}}=\sin{(\theta)}\) pois
    \(0\leqslant{\theta}\leqslant{\frac{\pi}{2}}\) e, assim,
    \(\sin{(\theta)}\geqslant{0}\).
\end{enumerate}

\begin{remark}
  Refaça os cálculos utilizando, desta vez, o Teorema de Green.
\end{remark}
