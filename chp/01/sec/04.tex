\section{Correção de um exercício da lista}\label{sec:correcao-de-um-exercício-da-lista}

\begin{exercise}
  The sample mean and sample variance of five data values are, respectively,
  \(\samplemean{x}=104\) and \(\samplevariance{x}=4\). If three of the data
  values are \(102,100,105\) what are the other two data values?
\end{exercise}

Suponha que
\(x_{1}^{0},x_{2}^{0},x_{3}^{0},\samplemean{x},\samplevariance{x}\in\reals\)
sejam constantes, onde \(\samplevariance{x}\geqslant{0}\), e que
\(x_{4},x_{5}\in\reals\) sejam números reais a serem encontrados de forma que
\[
  \left\{
    \begin{array}{rcl}
      \samplemean{x}
      &=&
      \dfrac{
        x_{1}^{0}+x_{2}^{0}+x_{3}^{0}+x_{4}+x_{5}
      }{5}
      \\
      &&
      \\
      \samplevariance{x}
      &=&
      \dfrac{
        (x_{1}^{0}-\samplemean{x})^{2}
        +
        (x_{2}^{0}-\samplemean{x})^{2}
        +
        (x_{3}^{0}-\samplemean{x})^{2}
        +
        (x_{4}-\samplemean{x})^{2}
        +
        (x_{5}-\samplemean{x})^{2}
      }{4}
    \end{array}
  \right.
\]
Observe, neste caso, que:
\begin{align}
  x_{4}+x_{5}
  &=
  5\samplemean{x}
  -
  \sum_{i=1}^{3}x_{i}^{0}
  \\
  (x_{4}-\samplemean{x})^{2}
  +
  (x_{5}-\samplemean{x})^{2}
  &=
  4\samplevariance{x}
  -
  \sum_{i=1}^{3}(x_{i}^{0}-\samplemean{x})^{2}
\end{align}
de onde segue que:
\[
  x_{4}^{2}+x_{5}^{2}
  -
  2\samplemean{x}(x_{4}+x_{5})
  +
  2\samplemean{x}^{2}
  =
  4\samplevariance{x}
  -
  \sum_{i=1}^{3}(x_{i}^{0})^{2}
  +
  2\samplemean{x}\sum_{i=1}^{3}x_{i}^{0}
  -
  3\samplemean{x}^{2}
  ,
\]
e, assim, que:
\[
  x_{4}^{2}+x_{5}^{2}
  -
  2
  \samplemean{x}
  \left(
    5\samplemean{x}
    -
    \sum_{i=1}^{3}x_{i}^{0}
  \right)
  +
  2\samplemean{x}^{2}
  =
  4\samplevariance{x}
  -
  \sum_{i=1}^{3}(x_{i}^{0})^{2}
  +
  2\samplemean{x}\sum_{i=1}^{3}x_{i}^{0}
  -
  3\samplemean{x}^{2}.
\]
Portanto, tem-se que:
\[
  5\samplemean{x}^{2}+4\samplevariance{x}-\sum_{i=1}^{3}(x_{i}^{0})^{2}
  =
  x_{4}^{2}+x_{5}^{2}
  \geqslant{0}.
\]
Isto mostra que os valores
\(x_{1}^{0},x_{2}^{0},x_{3}^{0},\samplemean{x},\samplestandarddeviation{x}\)
precisam satisfazer à relação:
\[
  \frac{\samplemean{x}^{2}}{a^{2}}
  +
  \frac{\samplevariance{x}}{b^{2}}
  \geqslant{1}
  ,
\]
onde
\[
  a=\frac{\sqrt{\sum_{i=1}^{3}(x_{i}^{0})^{2}}}{\sqrt{5}}
  \quad
  \text{e}
  \quad
  b=\frac{\sqrt{\sum_{i=1}^{3}(x_{i}^{0})^{2}}}{2}
  .
\]
Sabendo disso, você consegue agora escolher valores adequados?
